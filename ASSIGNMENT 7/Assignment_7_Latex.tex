\documentclass[11pt, a4paper]{article}
\usepackage{graphicx}
\usepackage{amsmath}
\usepackage{listings}
\usepackage{minted}
\usepackage{physics}
\usepackage{comment}

\title{\underline{\textit{\Large{Assignment 7}}}}

\author{\textit{ EE2703- APPLIED PROGRAMMING LANGUAGE - SUHAS C EE20B132 }}

\date{ Date - 08/04/22} 

\begin{document}
\maketitle

\section*{Abstract}
This week's assignment involves the analysis of lters using laplace transforms.
Python's symbolic solving library, sympy is a tool we use in the process to
handle our requirements in solving Modied Nodal Analysis equations. Besides
this the library also includes useful classes to handle the simulation and response
to inputs.
Coupled with scipy's signal module, we are able to analyse both High pass
and low pass lters, both second order, realised using a single op amp

\section*{Low Pass Filter}
The low pass filter that we use gives the following matrix equation after simplification of the modified nodal equations.
\[\begin{pmatrix} 0 & 0 & 1 & -\frac{1}{G} \\ -\frac{1}{1+sR_2C_2} & 1 & 0 & 0 \\ 0 & -G & G & 1 \\ -\frac{1}{R_1}-\frac{1}{R_2}-sC_1 & \frac{1}{R_2} & 0 & sC_1 \end{pmatrix}\begin{pmatrix} V_1 \\ V_p \\ V_m \\ V_o \end{pmatrix} = \begin{pmatrix} 0 \\ 0 \\ 0 \\ -V_i(s)/R_1 \end{pmatrix}\]

The python code snippet that declares the low pass function and solves the matrix equation to get the V matrix is as shown below:
%\begin{minted}{python3}
\begin{verbatim}
# Defifning Low Pass Filter
def lowpass(R1,R2,C1,C2,G,Vi):
    
    A=Matrix([[0,0,1,-1/G],                                 # Conductance Matrix
              [-1/(1+s*R2*C2),1,0,0],
              [0,-G,G,1], 
              [-1/R1-1/R2-s*C1,1/R2,0,s*C1]])
    
    b=Matrix([0,0,0,Vi/R1])                                 # Source Matrix

    V = A.inv()*b                                           # Variable Matrix
    return (A,b,V)
\end{verbatim}
%\end{minted}
The plot for the magnitude of the transfer function (magnitude bode plot) is as shown below:
\begin{figure}[!tbh]
   	\centering
   	\includegraphics[width=1.0\textwidth]{Figure0.png}
   	\label{fig:32}
   	\caption{Low pass filter Magnitude response}
   \end{figure}

\section*{High Pass Filter}
The high pass filter we use gives the following matrix equations after simplification of the modified nodal equations
\[\begin{pmatrix} 0 & 0 & 1 & -\frac{1}{G} \\ -\frac{-sR_3C_2}{1+sR_3C_2} & 1 & 0 & 0 \\ 0 & -G & G & 1 \\ -1-(sR_1C_1)-(sR_3C_2)) & sC_2R_1 & 0 & 1 \end{pmatrix}\begin{pmatrix} V_1 \\ V_p \\ V_m \\ V_o \end{pmatrix} = \begin{pmatrix} 0 \\ 0 \\ 0 \\ -V_i(s)sR_1C_1 \end{pmatrix}\]

The python code snippet that declares the high pass function and solves the matrix equation to get the V matrix is as shown below:

%\begin{minted}{python3}
\begin{verbatim}
# Defining High Pass Filter    
def highpass(R1,R3,C1,C2,G,Vi):                     
     
     A = Matrix([[0,0,1,-1/G],                              # Conductance Matrix
                 [0,G,-G,-1],
                 [s*C2*R3/(1+s*C2*R3),-1,0,0],
                 [(s*C2+s*C1+1/R1),-s*C2,0,-1/R1]])
                 
     b = Matrix([0,0,0,s*C1*Vi])                            # Source Matrix
     
     V = A.inv()*b
     return (A,b,V) 
\end{verbatim}
%\end{minted}
The plot for the magnitude of the transfer function (magnitude bode plot) is as shown below:
\begin{figure}[!tbh]
   	\centering
   	\includegraphics[width=1.0\textwidth]{Figure3.png}
   	\label{fig:32}
   	\caption{High pass filter Magnitude response}
   \end{figure}
\section*{Change of Format in Sympy Functions}
The sympy functions,that are expressed in terms of 's', must be converted to another form that is understood by sp.signal. This is done using the \textit{sympyToTrFn()} function.\\
The python code snippet is as shown:
%\begin{minted}{python3}
 \begin{verbatim}
# Function for converting SymPy symbols to Polynomials     
def Symbols_to_Polynomial(H):
	n,d=fraction(simplify(H))
	return (np.array(Poly(n,s).all_coeffs(),dtype=float),np.array(Poly(d,s).all_coeffs(),dtype=float))
\end{verbatim}	
%\end{minted}

\section*{Low Pass Filter Step Response}
In order to find the step response of the low pass filter, we need to assign $ Vi(s) = 1/s $. 
The python code below explains it. 
%\begin{minted}{python3}
\begin{verbatim}
# Question_1

A1,b1,V1 = lowpass(10000,10000,1e-9,1e-9,1.586,1)           # Finding the Tansfer Function of low pass filter
Nr_H1,Dr_H1 = Symbols_to_Polynomial(V1[3])
H1 = sp.lti(Nr_H1,Dr_H1)                                    # Converting Sympy Function to Signal Transfer Function
t  = np.linspace(0,0.01,50001)                              # Time steps for plotting the input and the output

Vi1 = np.ones(50001)                                        # Unit Step input function
Vi1[0] = 0
t,Vo1,svec = sp.lsim(H1,Vi1,t)                              # Finding the output in time domain

                   
py.figure(0)                                                 
py.plot(t,abs(Vo1),label = 'Output')                        # plotting the input and the output  
py.plot(t,Vi1,label = 'Input')                            
py.legend(loc = 'upper right')
py.title(r'Q_1:Unit Step Response of Low-Pass Filter')
py.grid(True)
py.savefig("suhas/figure0.png")
py.show()

\end{verbatim}
%\end{minted}
The plot for the step response of the low pass circuit is as shown:
\begin{figure}[!tbh]
   	\centering
   	\includegraphics[width=1.0\textwidth]{Figure1.png}
   	\label{fig:32}
   	\caption{System Response of Low Pass Filter with Decay = 0.5.}
   \end{figure}
\section*{Response to Sum Of Sinusoids}   
    When the input is a sum of sinusoids like,
\begin{equation*}
  V_{i}(t) = ( \ \sin(2000\pi t) + \cos(2*10^{6}\pi t) \ )u_{o}(t) \ Volts
  \end{equation*}
  Then the output response for the lowpass filter can easily be found as shown in the code snippet.
%  \begin{minted}{python3}
\begin{verbatim}
# Question_2 

Vi2 = np.sin(2*(10**3)*(np.pi)*t)+np.cos(2*(10**6)*(np.pi)*t)                            # Sinusoidally oscillating input to the Low pass filter
t,Vo2,svec = sp.lsim(H1,Vi2,t)                                                           # Convolution of the input with the Impulse Response

py.figure(1)                                                                             # plotting the input and the output
py.plot(t,Vi2,label = 'Input')
py.plot(t,-Vo2,label = 'Output')
py.legend(loc = 'upper right')
py.title(r'Q_2:Response for Sinusoidal Input for Low-Pass Filter')
py.grid(True)
py.savefig("suhas/figure1.png")
py.show()



\end{verbatim}
%  \end{minted}
The output response to the sum of sinusoids for the low pass filter is as shown:
\begin{figure}[!tbh]
   	\centering
   	\includegraphics[width=1.0\textwidth]{Figure2.png}
   	\label{fig:32}
   	\caption{Output for sum of sinusoids}
   \end{figure}
   \newpage
\section*{Response to Damped Sinusoids}
In this case we assign the input voltage as a damped sinusoid like,\\
Low frequency,
\begin{equation*}
    V_{i}(t) = e^{-500t}( \ \cos(2000\pi t) \ )u_{o}(t) \ Volts
\end{equation*}
High frequency,
\begin{equation*}
    V_{i}(t) = e^{-500t}( \ \cos(2*10^{6}\pi t) \ )u_{o}(t) \ Volts
\end{equation*}
The high frequency and low frequency plots of the input damped sinusoids are as shown:\newpage
\begin{figure}[!tbh]
   	\centering
   	\includegraphics[width=1.0\textwidth]{Figure4.png}
   	\label{fig:32}
   	\caption{High frequency damped sinusoid}
   \end{figure}
   
   Thus, the high-pass behaviour of the circuit is verified.  The change in the exponential would only affect the rate at which the sinusoid amplitude decays to zero.
The python code snippet to execute the above is as shown:
%\begin{minted}{python3}
\begin{verbatim}
# Question_3

A3,b3,V3 = highpass(10000,10000,1e-9,1e-9,1.586,1)                                       # Finding the Tansfer Function of high pass filter 
Vo3 = V3[3]
w  = py.logspace(0,8,801)                                                                # Range of Frequencies for the function to be plotted
ss = 1j*w
hf = lambdify(s,Vo3,'numpy')                                                             # converting the Sympy variable into python function
v  = hf(ss)

py.figure(2)
py.loglog(w,abs(v),lw=2,label = 'Magnitude Response of an High-Pass Filter')             # Plotting the Magnitude Response of High pass filter
py.title(r'Q_3:Magnitude Response of a High-Pass Filter')
py.grid(True)
py.savefig("suhas/figure2.png")
py.show()



# Question_4

t4 = np.linspace(0,1,100000)                                                             # Time steps for plotting the input and the output
Vi4 = (np.cos(2*(np.pi)*(10**4)*t4))*(np.exp(-5*t4))                                     # Sinusoudall Damped Oscillating input for high pass filter

Nr_H4,Dr_H4 = Symbols_to_Polynomial(Vo3)                                                 # Converting Sympy Variable to Signal Transfer Function
H4 = sp.lti(Nr_H4,Dr_H4)

t4,Vo4,svec = sp.lsim(H4,Vi4,t4)                                                         # Convolution of the input with the Impulse Response

py.figure(3) 
py.plot(t4,Vi4,label = 'Input freq = 10^4Hz')                                            # plotting the input and the output  
py.plot(t4,Vo4,label = 'Output')
py.legend(loc = 'upper right')
py.title(r'Q_2:Response for Damped-Sinusoidal Input for High-Pass Filter')
py.grid(True)
py.savefig("suhas/figure3.png")
py.show()
\end{verbatim}


\section*{Step Response of High Pass Filter}
In order to find the step response of the high pass filter, we need to assign $ Vi(s) = 1/s $. 
The python code below shows:-
%\begin{minted}{python3}
\begin{verbatim}
# Question_5

A5,b5,V5 = highpass(10000,10000,1e-9,1e-9,1.586,1/s)                                     # Finding the Step Response of the high pass filter
Vo5 = V5[3]
t5  = np.linspace(0,0.001,50001)                                                           # Time steps for plotting the input and the output

Vi5 = np.ones(50001)                                                                     # Unit Step input function                                                          
Vi5[0] = 0
t5,Vo5,svec = sp.lsim(H4,Vi5,t5)                                                           # Convolution of Unit step with impulse response of high pass filter

py.figure(4) 
py.plot(t5,abs(Vo5),color='black',label = 'Output')                                       # plotting the input and the output
py.plot(t5,Vi5,label = 'Input')
py.legend(loc = 'upper right')
py.title(r'Q_1:Unit Step Response of High-Pass Filter')
py.grid(True)
py.savefig("suhas/figure4.png")
py.show()
\end{verbatim}
%\end{minted}
The plot of the step response for a high pass filter is as shown:

The unit step response, as expected is high at t=0 when there is an abrupt
change in the input. Since there is no other change at large time values outside
the neighbourhood of 0, the Fourier transform of the unit step has high values
near 0 frequency, which the high pass filter attenuates.

\section*{Conclusion}
In conclusion, the sympy module has allowed us to analyse quite complicated
circuits by analytically solving their node equations. We then interpreted the solutions by plotting time domain responses using the signals toolbox. Thus,
sympy combined with the scipy.signal module is a very useful toolbox for ana-
lyzing complicated systems like the active lters in this assignment.

  
\end{document}
